\documentclass[12pt, a4paper]{report}
\usepackage{ucs}                            % позволяет использовать utf8x
\usepackage[utf8x]{inputenc}                % кодовая страница документа
\usepackage[english, russian]{babel}                 % локализация и переносы
\usepackage{cmap}                           % русский поиск в pdf
\usepackage[usenames,dvipsnames]{color}     % названия цветов

% -----------------------------------------------------------------------------------------------------------------------------------------------------------------------


% Настрока геометрии страницы
\usepackage{geometry} % задаёт поля 
\geometry{left=3cm} % левое — 3 см
\geometry{right= 1.5cm} % правое — 1,5 см
\geometry{top=2cm} % верхнее — 2 см
\geometry{bottom=2cm} % нижнее — 2 см

% -----------------------------------------------------------------------------------------------------------------------------------------------------------------------


% Настрока текста
\usepackage{setspace} \onehalfspacing   % задаёт «полуторный» межстрочный интервал 
\usepackage{indentfirst}                % русский стиль: отступ первого абзаца раздела
\usepackage{misccorr}                   % точка в номерах заголовков
\usepackage{multirow}                   % Аналог multicolumn для строк
\usepackage{ltxtable}                   % Микс tabularx и longtable
\usepackage{paralist}                   % Списки с отступом только в первой строчке
\usepackage[perpage]{footmisc}          % Нумерация сносок на каждой странице с 1
%\pagenumbering{gobble}                 %Когда включено, отключает нумерацию страниц

% -----------------------------------------------------------------------------------------------------------------------------------------------------------------------


% Колонтитулов страницы

% Подключим пакет fancyhdr, предназначенный для оформления верхних и нижних колонтитулов страницы, и поотключаем в нем всякие умолчания. 
% Страница становится абсолютно чистой, за исключением номера (\thepage) в правом углу верхнего колонтитула.
\usepackage{fancyhdr}
\pagestyle{fancy}
\fancyhf{}
\fancyhead[R]{\thepage}
\fancyheadoffset{0mm}
\fancyfootoffset{0mm}
\setlength{\headheight}{17pt}
\renewcommand{\headrulewidth}{0pt}
\renewcommand{\footrulewidth}{0pt}
\fancypagestyle{plain}{
	\fancyhf{}
	\rhead{\thepage}}
\setcounter{page}{1} % начать нумерацию страниц с №1

% -----------------------------------------------------------------------------------------------------------------------------------------------------------------------


% Заголовки
\usepackage{titlesec}

\makeatletter
\renewcommand{\@chapapp}{Задание} % изменяем глава на задание
\makeatother

% \setcounter{secnumdepth}{0}

% первый — уровень настраиваемого заголовка (например, chapter или section);
% второй необязательный — форма заголовка;
% третий параметр — команды, вызывающиеся перед печатью всего заголовка;
% четвертый параметр — оформление метки;
% пятый параметр — расстояние между меткой и текстом заголовка
% шестой параметр — команды, вызывающиеся перед печатью текста заголовка;
% седьмой необязательный — команды, вызывающиеся после печати текста заголовка.
\titleformat{\chapter}[display]
{\filcenter}
{\Large\chaptertitlename\ №\thechapter}
{1em}
{\bfseries}{}

\titleformat{\section}
{\large\bfseries}
{}
{0em}{}

\titleformat{\subsection}
{\normalsize\bfseries}
{\thesubsection}
{0em}{}

% Настройка вертикальных и горизонтальных отступов
\titlespacing*{\chapter}{0pt}{-30pt}{8pt}
\titlespacing*{\section}{\parindent}{*4}{*4}
\titlespacing*{\subsection}{\parindent}{*4}{*4}

% -----------------------------------------------------------------------------------------------------------------------------------------------------------------------


% Графика
\usepackage{graphicx}                               % Работа с графикой \includegraphics{}
\graphicspath{ {./src/} }

\usepackage{psfrag}                                 % Замена тагов на eps картинкаx
\usepackage[export]{adjustbox}                      % Обрезка, подгонка картинок
\usepackage{tikz}                                   % Встроенная рисовалка
\usetikzlibrary{calc,shapes,arrows,chains,fit}      % Библиотеки для рисовалки.
\usepackage{pgfplots}                               % Встроенная чертилка. Работает внутри рисовалки.
\usepackage{pgfplotstable}                          % Черчение данных из файлов
\usepgfplotslibrary{groupplots}                     % Множественные графики. График над графиком.
\usetikzlibrary{external}                           % Ускоряет сборку документов с картинками (?)

% -----------------------------------------------------------------------------------------------------------------------------------------------------------------------


% Таблицы
\usepackage{wrapfig}                                % Обтекание фигур (таблиц, картинок и прочего)
\usepackage{longtable}                              % Многостраничные таблицы
\usepackage{booktabs}                               % полезная либа для таблиц
\usepackage{hhline}
\usepackage{cellspace}
\usepackage{boldline}

% -----------------------------------------------------------------------------------------------------------------------------------------------------------------------


% Математические пакеты
\usepackage{amssymb,amsfonts,amsmath,amsthm}    % математические дополнения от АМС
\usepackage{mathtools}                          % Прямое указание типа дробей и прочее
\usepackage{physics}                            % полезный пакет с макетами формул, сокращает код
\usepackage{cancel}                             % позволяет использовать зачеркивание
\usepackage{stackrel}

% -----------------------------------------------------------------------------------------------------------------------------------------------------------------------

% изменение нумерации формул, таблиц, фигур
% обязательно нужно перед этим включить пакет amsmath
\numberwithin{equation}{section}
\numberwithin{table}{section}
\numberwithin{figure}{section}

% -----------------------------------------------------------------------------------------------------------------------------------------------------------------------


% Подписи под изображениями и таблицами
\usepackage[]{caption}
\usepackage{subcaption}
% настраивается новый пользовательский формат оформления подписи
% Его первый параметр #1 — это стандартный текст метки 
% второй параметр #2 — номер рисунка/таблицы.
\DeclareCaptionLabelFormat{gostfigure}{Рисунок #2}
\DeclareCaptionLabelFormat{gosttable}{Таблица #2}
\DeclareCaptionLabelSeparator{defffis}{~---~}              % настраивает разделитель между меткой подписи и непосредственно её текстом
\captionsetup{labelsep=defffis}
\captionsetup[figure]{labelformat=gostfigure}
\captionsetup[table]{labelformat=gosttable}
\renewcommand{\thesubfigure}{\asbuk{subfigure}}         % кириллическое представление счетчиков

% -----------------------------------------------------------------------------------------------------------------------------------------------------------------------


% Пакеты ссылок
% \usepackage{hyperref}                       % Поддержка ссылок в PDF. Делает все ссылки в PDF рабочими
\usepackage[russian]{cleveref}              % Умные ссылки -- \cref{fig:123} ссылается на картинку, "рис. 123". Умеет ссылки на список, причем умные

\begin {document}
привет
\end {document}