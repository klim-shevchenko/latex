\section{Анализ предметной области}
\subsection{История первых игр что стали прародителями RPG-жанра}

Разговор о самых первых компьютерных ролевых играх требует двух важных оговорок. В середине 70-х компьютеры еще не были персональными и представляли собой огромные машины, занимавшие порой отдельные помещения, и были оборудованы подключенными в единую систему терминалами. Доступ к ним был у немногих избранных, а единственными из них, кому могла прийти в голову делать для этих компьютеров игры, были студенты технических университетов. Соответственно, ни у одной из созданных этими первопроходцами игр не было никаких шансов на коммерческий релиз. 

Сейчас уже сложно установить, какой была первая видеоигра, которую можно было бы отнести к жанру RPG. Многие из них безнадежно сгинули в пучине истории. Например, теоретически претендующая на почетное первенство игра под названием m199h,  созданная в 1974-м в Университете Иллинойса почти сразу после выхода первой редакции DnD, была попросту удалена кем-то из преподавателей — компьютеры ведь созданы для обучения, а не для игрушек. Зато вот появившаяся примерно тогда же The Dungeon сохранилась до наших дней. Она также известна как pedit5 — это название исполняемого файла, который юный разработчик Расти Рутерфорд замаскировал под учебный. От удаления смекалочка игру не спасла, но исходный код уцелел, и сыграть в нее можно даже сегодня.

Привыкшие к современным RPG геймеры от увиденного могут испытать культурный шок. Но даже по меркам середины 70-х эти игры казались примитивными. Причем не в сравнении с другими жанрами видеоигр, а в сравнении со все теми же настолками. Если за игровым столом в компании друзей подробности приключения и игровой мир в деталях рисовало воображение игроков, и лишь оно ограничивало пределы игры, то скудная презентация этих ранних видеоигровых экспериментов и близко не давала такого опыта. Тем более речи не шло ни о каком серьезном отыгрыше роли и глубоком нарративе, к которым нас приучили вышедшие многим позже шедевры жанра. Чтобы называться компьютерной RPG, в те годы игре достаточно было обладать какой-никакой системой прокачки, да давать возможность отыгрывать в бою воина или мага.

В том, что касается сюжета, диалогов и повествования в целом для  жанра гораздо больше сделала игра, которую даже в 1976 году никому не пришло бы в голову назвать ролевой — Colossal Cave Adventure. По сути это прабабушка всех текстоцентричных игр: от RPG с объемными диалогами до интерактивных сериалов и даже визуальных новелл. Она могла бы быть стандартной адвенчурой про исследователя пещер, пытающегося найти сокровища в лабиринте, каких было немало. Вот только ее создатель Уилл Кроутер решил полностью отказаться от графики, забив экран монитора детальным описанием окружения, и тем самым не только вновь отдал бремя проработки деталей на откуп фантазии игрока, но и легитимировал текстовый нарратив для всех будущих разработчиков. 

\subsection{Первые популярные RPG-игры}
Akalabeth стала основой всех будущих dungeon crawler — игр с упором на исследование подземелий. Сохранив геймплейную основу ранних RPG — зачистку подземелий, классы и прокачку — она впервые объединила вид от первого лица при прохождении уровня и вид сверху при перемещении по миру. В игре присутствовала и механика провизии, за объемом которой нужно было постоянно следить, и проработанная система заклинаний, применение которых вызывало подчас совершенно неожиданные последствия. Фантазии, смелости и амбиций автору было не занимать. Последнее особенно подчеркивает существование в мире игры персонажа по имени Lord British, от которого игрок и получал все задания. Разработка Гэрриота оказалась настолько нетривиальной, что ей заинтересовался крупный издатель. Смешные по сегодняшним меркам продажи в 30 тысяч копий обрекли Akalabeth  на сиквел, а ее автора — на профессию игрового разработчика. Так началась многолетняя история одной величайших игровых серий прошлого — Ultima.

Благодаря развитию технологий, большему бюджету и поддержке издателя Гэрриот сумел в кратчайшие сроки значительно улучшить техническую составляющую игры — вышедшая спустя год Ultima обзавелась тайловой графикой, а для управления персонажем больше не нужно было вводить текстовые команды — достаточно было нажатия на кнопки со стрелочками. Но больше всего аудиторию поразил небывалый размах приключения: мало того, что игровой мир стал куда более объемным, а благодаря современной графике выглядел реальнее, чем когда-либо, так еще и повествование охватывало аж три временные эпохи.

Между тем игры Гэрриота обрели достойную конкуренцию в лице не менее значительной для жанра серии Wizardry. Созданная в 1981 году командой Sir-Tech Software в лице Эндрю Гринберга и Роберта Вудхеда, она не хватала звезд с неба ни в плане графики, ни в плане сюжета, зато геймплейно была глубже и проработаннее любой другой CRPG. Если Гэрриот ориентировался на посиделки в DnD и старался перенести на экран волшебный антураж, рисуемый воображением, то разработчики Wizardry ставили себе цель вывести на новый уровень игры с мейнфреймов, в которые залипали в студенческие годы. Для них на первом месте была механика. 
Весь игровой мир изображался в маленьком квадратике в углу экрана, большую же его часть заполняла важная для прохождения информация — очки здоровья и классы бойцов, список заклинаний, данные о противнике. При создании каждого из шести играбельных персонажей можно было не только выбрать расу, класс и распределить очки характеристик, но и прописать героям мировоззрение, влияющее на дальнейшую прокачку. Таким образом Wizardry еще и стала первой партийной RPG в истории, так что корни Baldur’s Gate, Icewind Dale и даже Divinity: Original Sin растут именно отсюда. Боевую систему сдобрили обширной системой магии, среди которой было место как прямо атакующим заклинаниям, так и различным дебаффам. А еще разработка Sir-Tech была беспощадно сложной: подобно Rogue в случае смерти партии игроку ничего не оставалось, кроме как начать с нуля

\subsection{Япония и её JRPG}
В 1986 году отобранная по конкурсу компанией Enix команда молодых и амбициозных японских технарей во главе с Юдзи Хории разработала и выпустила первую в истории JRPG под названием Dragon Quest. Именно эта игра сформировала основные правила поджанра на десятилетия вперед: вид сверху, более-менее свободное исследование огромного мира, состоящего из квадратных тайлов, случайные встречи, пошаговый бой, отдельное окно для сражений с изображением противника и списком возможных действий, а также большой акцент на линейное повествование с неизменными тропами: древнее зло, магические артефакты, спасение принцессы... Здесь же любители RPG впервые столкнулись с около-анимешной эстетикой, за которую отвечал специально привлеченный в качестве художника известный мангака Акира Торияма.

На старте 1987 года компания Square, обреченная в будущем стать второй (или первой?) половинкой Enix, выпустила на японский рынок игру, с которой началась история длиною в жизнь. И если Dragon Quest изобрела жанр, то синонимом JRPG стало имя Final Fantasy.
И ведь, казалось бы, на первый взгляд игра Хиронобу Сакагучи не сильно отличалась от своей предшественницы из Enix. С геймплейной точки зрения ключевым изменением стала система классов — игрок мог по желанию сделать любого из четверки героев воином, вором, монахом или магом одной из школ. Но главное, чем брала Final Fantasy, — небывалой амбициозностью во всем. В ее мире присутствовали и элементы стимпанка, и научная фантастика, и петля времени, которую бравым героям необходимо было разомкнуть… Постановка также была яркой и необычной для своего времени: например, представляющую игру заставку и титры игрок видел лишь после выполнения первого квеста — прием, активно взятый на вооружение современными разработчиками.

\subsection{Популярные RPG студии SSI}
Главным же поставщиком RPG на грани десятилетий стала компания SSI. В 1988 году ее президент Джоэл Биллингс ввязался в крупнейшую авантюру своей жизни: в жесточайшей конкуренции за огромные деньги выкупил официальную лицензию на создание игр по обновленной редакции легендарной настолки Advanced Dungeons and Dragons. В следующие пять лет SSI выпустила целых 12 компьютерных ролевых игр, вошедших в историю под общим именем Gold Box. Откровенно говоря, большая их часть не изобретала велосипеда. Они лишь довели знакомую жанровую схему предшественниц до совершенства и сопроводили ее достаточным количеством оригинального контента — врагов, квестов, оружия, элементов окружения. Из важных деталей стоит отметить возможность избежать сражения с врагом путем дипломатии (для этого необходимо было выбрать правильный тон разговора) и функцию быстрого перемещения с помощью раскинувшейся по игровому миру сети телепортов. 
Лицензия DnD распространялась и на использование различных сеттингов настолки, поэтому местом действия игр могли стать как «Забытые Королевства», так и вселенная «Драконьего Копья». Первоисточник даровал разработчикам не только готовую механику, но и проработанную мифологию. Такой мощный фундамент позволял стабильно выпускать новинки раз в несколько месяцев. Наладив потоковое производство, SSI превратила создание ролевых игр в индустрию. Вскоре каталог компании пополнили и игры сторонних студий, разработанные по драгоценной лицензии, в числе которых была, например, популярная трилогия Eye of the Beholder от Westwood Studios.

Два релиза из коллекции Gold Box заслуживают отдельного внимания. Во-первых, это выпущенная в 1993 году Forgotten Realms: Unlimited Adventures, которая технически являлась не игрой, а набором инструментов для создания собственных приключений, основанных на ADnD. Некоторые безумные традиционалисты от мира ролевых игр до сих пор пользуются этой программой для разработки нового контента, а в 90-е она устроила настоящий переворот в фанатских кругах и предопределила формирование сообщества моддеров. Не менее важным событием стал выход в 1991-м Neverwinter Nights. Сейчас эту игру затмил другой релиз под таким же названием, случившийся уже в следующем веке, но в истории индустрии она останется навсегда. 

Она не выделялась на фоне других игр SSI ни внешним видом, ни ролевой системой, но один важный нюанс делал ее особенной: Neverwinter Nights стала первой полноценной графической MMORPG. Ее серверы вмещали до 50 игроков одновременно, общая же аудитория исчислялась сотнями тысяч. Фанаты объединялись в гильдии, вступали в виртуальные конфликты и проводили в онлайне массовые сходки. Интерес к Neverwinter Nights не увядал вплоть до ее закрытия в 1997 году, а ее влияние на дальнейшее развитие индустрии неоценимо. 
