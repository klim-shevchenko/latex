\section{Рабочий проект}
\subsection{Классы, используемые при разработке приложения}

\subsubsection{Класс Sprite}

Класс Sprite относится к rpg и используется для отображения изображения на холсте.

Описание методов класса Sprite представлено в таблице \ref{table:sprite_methods}.
\renewcommand{\arraystretch}{0.8} % уменьшение расстояний до сетки таблицы
\begin{xltabular}{\textwidth}{|X|X|}
	\caption{Методы класса Sprite}\label{table:sprite_methods} \\
	\hline \centrow
	Название метода & \centrow  Описание метода \\
	\hline \centrow 1 & \centrow 2 \\ \hline
	\endfirsthead
	\continuecaption{Продолжение таблицы \ref{table:sprite_methods}}\\
	\hline \centrow 1 & \centrow 2 \\ \hline
	\finishhead
	set\_tag(self, tag) & Устанавливает тег спрайта, параметр tag: тег спрайта. \\
	\hline
	set\_z(self, z) & Устанавливает z-координату спрайта, параметр z: координата z. \\
	\hline
	get\_tag(self) & Возвращает тег спрайта. \\
	\hline
	set\_coords(self, new\_x, new\_y) & Обновляет координаты спрайта, параметр new\_x: координата x, параметр new\_y: координата y. \\
	\hline
	update(self) & Обновляет анимацию спрайта. \\
	\hline
\end{xltabular}

\subsubsection{Класс Animation}

Класс Animation относится к rpg и предназначен для отображения набора изображений, как анимации на холсте.

Описание методов класса Animation представлено в таблице \ref{table:animation_methods}.

\begin{xltabular}{\textwidth}{|X|X|}
	\caption{Методы класса Animation}\label{table:animation_methods} \\
	\hline \centrow
	Название метода & \centrow  Описание метода \\
	\hline \centrow 1 & \centrow 2 \\ \hline
	\endfirsthead
	\continuecaption{Продолжение таблицы \ref{table:animation_methods}}\\
	\hline \centrow 1 & \centrow 2 \\ \hline
	\finishhead
	update(self) & Меняет текущее изображение в списке изображений. \\
	\hline
\end{xltabular}

\subsubsection{Класс Graphics}

Класс Graphics относится к rpg и предназначен для работы графической системы в платформе.

Описание методов класса Graphics представлено в таблице \ref{table:Graphics_methods}.

\begin{xltabular}{\textwidth}{|X|X|}
	\caption{Методы класса Graphics}\label{table:Graphics_methods} \\
	\hline \centrow
	Название метода & \centrow  Описание метода \\
	\hline \centrow 1 & \centrow 2 \\ \hline
	\endfirsthead
	\continuecaption{Продолжение таблицы \ref{table:Graphics_methods}}\\
	\hline \centrow 1 & \centrow 2 \\ \hline
	\finishhead
	add\_sprite(self, sprite, x, y, z, **kwargs) & Добавляет спрайт на Canvas, параметр sprite: спрайт, параметр x: координата x, параметр y: координата y, параметр z: координата z, параметр kwargs: параметры относящиеся к конкретному изображению в tkinter. \\
	\hline
	update(self) & Перерисовывает все спрайты. \\
	\hline
	change\_sprite(self, sprite, new\_sprite) & Меняет спрайт на новый в Canvas, параметр sprite: экземпляр спрайта, параметр new\_sprite: новый спрайт. \\
	\hline
	delete\_sprite(self, sprite) & Удаляет спрайт с Canvas, параметр sprite: экземпляр спрайта. \\
	\hline
	clear\_all(self) & Удаляет все спрайты с Canvas. \\
	\hline
\end{xltabular}

\subsubsection{Класс Rectangle}

Класс Rectangle относится к rpg и предназначен для создания прямоугольника, который нужен для объектов и зон.

Описание методов класса Rectangle представлено в таблице \ref{table:Rectangle_methods}.

\begin{xltabular}{\textwidth}{|X|X|}
	\caption{Методы класса Rectangle}\label{table:Rectangle_methods} \\
	\hline \centrow
	Название метода & \centrow  Описание метода \\
	\hline \centrow 1 & \centrow 2 \\ \hline
	\endfirsthead
	\continuecaption{Продолжение таблицы \ref{table:Rectangle_methods}}\\
	\hline \centrow 1 & \centrow 2 \\ \hline
	\finishhead
	is\_in(self, rect) &Проверяет, входит ли прямоугольник self в прямоугольник rect, параметр rect: прямоугольник. \\
	\hline
	is\_point\_inside(self, target\_x, target\_y) & Проверяет, входит ли точка (x, y) в данный прямоугольник, параметр target\_x: точка x, параметр target\_y: точка y. \\
	\hline
\end{xltabular}

\subsubsection{Класс Object}

Класс Object относится к rpg и предназначен для инициализации объектов.

Описание методов класса Object представлено в таблице \ref{table:Object_methods}.

\begin{xltabular}{\textwidth}{|X|X|}
	\caption{Методы класса Object}\label{table:Object_methods} \\
	\hline \centrow
	Название метода & \centrow  Описание метода \\
	\hline \centrow 1 & \centrow 2 \\ \hline
	\endfirsthead
	\continuecaption{Продолжение таблицы \ref{table:Object_methods}}\\
	\hline \centrow 1 & \centrow 2 \\ \hline
	\finishhead
	set\_state(self, state\_name) & Меняет текущее состояние объекта, параметр state\_name: новое состояние. \\
	\hline
	actor\_in(self, actor) & Вызывается когда персонаж входит внутрь объекта, параметр actor: персонаж входящий в объект. \\
	\hline
	update(self) & Обновляетинформацию об объекте. \\
	\hline
\end{xltabular}

\subsubsection{Класс Portal}

Класс Portal относится к rpg и предназначен для объектов используемых для переходов между зонами.

Описание методов класса Portal представлено в таблице \ref{table:Portal_methods}.

\begin{xltabular}{\textwidth}{|X|X|}
	\caption{Методы класса Portal}\label{table:Portal_methods} \\
	\hline \centrow
	Название метода & \centrow  Описание метода \\
	\hline \centrow 1 & \centrow 2 \\ \hline
	\endfirsthead
	\continuecaption{Продолжение таблицы \ref{table:Portal_methods}}\\
	\hline \centrow 1 & \centrow 2 \\ \hline
	\finishhead
	actor\_in(self, actor) & Проверяет находится ли персонаж внутри портала, параметр actor: проверяемый персонаж. \\
	\hline
\end{xltabular}

\subsubsection{Класс Actor}

Класс Actor относится к rpg и предназначен для инициализации персонажей, а так для хранения базовых механик передвижения.

Описание методов класса Actor представлено в таблице \ref{table:Actor_methods}.

\begin{xltabular}{\textwidth}{|X|X|}
	\caption{Методы класса Actor}\label{table:Actor_methods} \\
	\hline \centrow
	Название метода & \centrow  Описание метода \\
	\hline \centrow 1 & \centrow 2 \\ \hline
	\endfirsthead
	\continuecaption{Продолжение таблицы \ref{table:Actor_methods}}\\
	\hline \centrow 1 & \centrow 2 \\ \hline
	\finishhead
	update(self) & Изменяет координаты и состояние персонажа. \\
	\hline
	search\_position(self, new\_x, new\_y) & Изменяет направление движения у персонажа, параметр new\_x: координата новой точки x, параметр new\_y: координата новой точки y. \\
	\hline
	stop\_move(self) & Останавливает движение персонажа. \\
	\hline
\end{xltabular}

\subsubsection{Класс Adnd\_actor}

Класс Adnd\_actor относится к rpg и предназначен для создания персонажа игры с механиками взаимодействия с другими персонажами.

Описание методов класса Adnd\_actor представлено в таблице \ref{table:Adnd_actor_methods}.

\begin{xltabular}{\textwidth}{|X|X|}
	\caption{Методы класса Adnd\_actor}\label{table:Adnd_actor_methods} \\
	\hline \centrow
	Название метода & \centrow  Описание метода \\
	\hline \centrow 1 & \centrow 2 \\ \hline
	\endfirsthead
	\continuecaption{Продолжение таблицы \ref{table:Adnd_actor_methods}}\\
	\hline \centrow 1 & \centrow 2 \\ \hline
	\finishhead
	click(self) & Вызывается при клике на персонажа. \\
	\hline
	attack(self, actor) & Совершает атаку по actor, параметр actor: персонаж, которого атакуют. \\
	\hline
	update(self) & Обновляет состояние персонажа. \\
	\hline
\end{xltabular}

\subsubsection{Класс Area}

Класс Area относится к rpg и предназначен для создания зоны, хранящей в себе определённое количество объектов, изображений.

Описание методов класса Area представлено в таблице \ref{table:Area_methods}.

\begin{xltabular}{\textwidth}{|X|X|}
	\caption{Методы класса Area}\label{table:Area_methods} \\
	\hline \centrow
	Название метода & \centrow  Описание метода \\
	\hline \centrow 1 & \centrow 2 \\ \hline
	\endfirsthead
	\continuecaption{Продолжение таблицы \ref{table:Area_methods}}\\
	Название метода & \centrow  Описание метода \\
	\hline \centrow 1 & \centrow 2 \\ \hline
	\finishhead
	add\_sprite(self, sprite, x, y, z) & Добавляет спрайт в зону, параметр sprite: экземпляр спрайта, параметр x: коодрината x, параметр y: коодрината y, параметр z: коодрината z. \\
	\hline
	add\_object(self, obj, x, y, z) & Добавляет объект в зону, параметр obj: объект, параметр x: коодрината x, параметр y: коодрината y, параметр z: коодрината z. \\
	\hline
	remove\_object(self, obj) & Удаляет объект из зоны. \\
	\hline
	load\_sprites(self) & Загружает все спрайты зоны. \\
	\hline
	add\_rect(self, rec) &  Добавляет прямоугольник в зону, параметр rec: прямоугольник. \\
	\hline
	entry\_script(self) & Запускается, когда команда входит в зону. \\
	\hline
	exit\_script(self) & Запускается, когда команда выходит из зоны. \\
	\hline
	update(self) & Изменяет и проверяет изменение всех объектов в зоне. \\
	\hline
\end{xltabular}

\subsubsection{Класс Game}

Класс Game относится к rpg и предназначен для управления игровой системой.

Описание методов класса Game представлено в таблице \ref{table:Game_methods}.

\begin{xltabular}{\textwidth}{|X|X|}
	\caption{Методы класса Game}\label{table:Game_methods} \\
	\hline \centrow
	Название метода & \centrow  Описание метода \\
	\hline \centrow 1 & \centrow 2 \\ \hline
	\endfirsthead
	\continuecaption{Продолжение таблицы \ref{table:Game_methods}}\\
	Название метода & \centrow  Описание метода \\
	\hline \centrow 1 & \centrow 2 \\ \hline
	\finishhead
	new\_area(self, name, area) & Добавляет новую зону в список, параметр name: имя зоны, параметр area: класс area. \\
	\hline
	set\_area(self, name) & Устанавливает текущую зону, загружает графику зоны, параметр name: имя зоны. \\
	\hline
	new\_actor(self, name, **params) & Создаёт класс, потомок от Actor и создаёт поле из параметров, и установление их в начальные значения,
	параметр name: название нового класса, параметр params: поля нового класса. \\
	\hline
	add\_pc\_to\_team(self, pc) & Добавляет персонажа в команду, параметр pc: персонаж, которого нужно добавить в команду. \\
	\hline
	remove\_pc\_from\_team(self, pc) &  Удаляет персонажа из команды, параметр pc: персонаж, которого нужно удалить. \\
	\hline
	start\_script(self, script\_function, script\_name, *args) & Запускает сценарий в отдельном потоке с возможностью остановки и передачи аргументов,
	параметр script\_function: Функция, содержащая код сценария,
	параметр script\_name: Имя сценария,
	параметр args: Дополнительные аргументы, которые нужно передать в сценарий.. \\
	\hline
	stop\_script(self, script\_name) & Останавливает сценарий по имени, параметр script\_name: имя сценария, который нужно остановить. \\
	\hline
	set\_team(self, x, y, z) & Устанавливает координаты персонажей команды, параметр x: координата x, параметр y: координата y, параметр z: координата z. \\
	\hline
	update(self) & Вызывается в таймере для обновления всех переменных в текущей зоне. \\
	\hline
	mouse\_left\_click(self, event) & Обрабатывает клик мыши, параметр event: клим мыши. \\
	\hline
	timer(self) & Таймер дожен вызывать метод update постоянно. \\
	\hline
\end{xltabular}

\subsubsection{Модуль Grunt}

Модуль Grunt относится к примеру работы платформы и предназначен для создания персонажа орка.

Описание методов модуля Grunt представлено в таблице \ref{table:Grunt_methods}.

\begin{xltabular}{\textwidth}{|X|X|}
	\caption{Методы модуля Grunt}\label{table:Grunt_methods} \\
	\hline \centrow
	Название метода & \centrow  Описание метода \\
	\hline \centrow 1 & \centrow 2 \\ \hline
	\endfirsthead
	\continuecaption{Продолжение таблицы \ref{table:Grunt_methods}}\\
	\hline \centrow 1 & \centrow 2 \\ \hline
	\finishhead
	new\_actor('Grunt', category='enemy', damage=10, hp=10, strange=5, wizdom=10, name='Grunt',
	states=\{\}) & Создаёт персонажа орка. \\
	\hline
\end{xltabular}

\subsubsection{Модуль Mage}

Модуль Mage относится к примеру работы платформы и предназначен для создания персонажа мага.

Описание методов модуля Mage представлено в таблице \ref{table:Mage_methods}.

\begin{xltabular}{\textwidth}{|X|X|}
	\caption{Методы модуля Mage}\label{table:Mage_methods} \\
	\hline \centrow
	Название метода & \centrow  Описание метода \\
	\hline \centrow 1 & \centrow 2 \\ \hline
	\endfirsthead
	\continuecaption{Продолжение таблицы \ref{table:Mage_methods}}\\
	\hline \centrow 1 & \centrow 2 \\ \hline
	\finishhead
	new\_actor('Mage', category='pc', damage=10, hp=10, strange=5, wizdom=10, name='Mage',
	states=\{\}) & Создаёт персонажа мага. \\
	\hline
\end{xltabular}

\subsubsection{Модуль Footman}

Модуль Footman относится к примеру работы платформы и предназначен для создания персонажа рыцаря.

Описание методов модуля Grunt представлено в таблице \ref{table:Footman_methods}.

\begin{xltabular}{\textwidth}{|X|X|}
	\caption{Методы модуля Footman}\label{table:Footman_methods} \\
	\hline \centrow
	Название метода & \centrow  Описание метода \\
	\hline \centrow 1 & \centrow 2 \\ \hline
	\endfirsthead
	\continuecaption{Продолжение таблицы \ref{table:Footman_methods}}\\
	\hline \centrow 1 & \centrow 2 \\ \hline
	\finishhead
	new\_actor('Footman', category='npc', damage=10, hp=10, strange=5, wizdom=10, name='Footman',
	states=\{\}) & Создаёт персонажа рыцаря. \\
	\hline
\end{xltabular}

\subsubsection{Модуль Village}

Модуль Village относится к примеру работы платформы и предназначен для создания зоны деревня.

Описание методов модуля Village представлено в таблице \ref{table:Village_methods}.

\begin{xltabular}{\textwidth}{|X|X|}
	\caption{Методы модуля Village}\label{table:Village_methods} \\
	\hline \centrow
	Название метода & \centrow  Описание метода \\
	\hline \centrow 1 & \centrow 2 \\ \hline
	\endfirsthead
	\continuecaption{Продолжение таблицы \ref{table:Village_methods}}\\
	\hline \centrow 1 & \centrow 2 \\ \hline
	\finishhead
	\_\_init\_\_(self) & Инициализирует зону Village. \\
	\hline
\end{xltabular}

\subsubsection{Модуль Ruins}

Модуль Ruins относится к примеру работы платформы и предназначен для создания зоны деревня.

Описание методов модуля Ruins представлено в таблице \ref{table:Ruins_methods}.

\begin{xltabular}{\textwidth}{|X|X|}
	\caption{Методы модуля Ruins}\label{table:Ruins_methods} \\
	\hline \centrow
	Название метода & \centrow  Описание метода \\
	\hline \centrow 1 & \centrow 2 \\ \hline
	\endfirsthead
	\continuecaption{Продолжение таблицы \ref{table:Ruins_methods}}\\
	\hline \centrow 1 & \centrow 2 \\ \hline
	\finishhead
	\_\_init\_\_(self) & Инициализирует зону Ruins. \\
	\hline
	walk(self, step\_x, step\_y, actor) & Сценарий для движения персонажа, параметр step\_x: шаг движения x, параметр step\_y: шаг движения y. \\
	\hline
	ai(self, actor) & Сценарий для противников, параметр step\_x: размер шага x до персонажа игрока, параметр step\_y: размер шага x до персонажа игрока, параметр actor: персонаж противник. \\
	\hline
\end{xltabular}

\subsubsection{Модуль bggame}

Модуль bggame относится к примеру работы платформы и предназначен для создания экземпляра игры.

Описание методов модуля bggame представлено в таблице \ref{table:bggame_methods}.

\begin{xltabular}{\textwidth}{|X|X|}
	\caption{Методы модуля bggame}\label{table:bggame_methods} \\
	\hline \centrow
	Название метода & \centrow  Описание метода \\
	\hline \centrow 1 & \centrow 2 \\ \hline
	\endfirsthead
	\continuecaption{Продолжение таблицы \ref{table:bggame_methods}}\\
	\hline \centrow 1 & \centrow 2 \\ \hline
	\finishhead
	\_\_init\_\_(self, canvas, window, **params) & Инициализирует игру bggame, параметр canvas: класс графической системы, параметр window: окно на которое будет выводится игра. \\
	\hline
\end{xltabular}

\subsubsection{Модуль main}

Модуль main относится к примеру работы платформы и предназначен для создания экземпляра игры.

Описание методов модуля main представлено в таблице \ref{table:main_methods}.

\begin{xltabular}{\textwidth}{|X|X|}
	\caption{Методы модуля main}\label{table:main_methods} \\
	\hline \centrow
	Название метода & \centrow  Описание метода \\
	\hline \centrow 1 & \centrow 2 \\ \hline
	\endfirsthead
	\continuecaption{Продолжение таблицы \ref{table:main_methods}}\\
	\hline \centrow 1 & \centrow 2 \\ \hline
	\finishhead
	root = tk.Tk() & Создаёт окно tkinter. \\
	\hline
	root.geometry('1500x1500') & Указывает размер окну tkinter. \\
	\hline
	canvas = Graphics(root, width=1500, height=1500) & Создание экземпляра класса graphics, который будет взаимодействовать с окном. \\
	\hline
	Graphics.canvas = canvas & Присваивание canvas в статическое поле. \\
	\hline
	BaldursGame(canvas, root) & Создание новой игры BaldursGame. \\
	\hline
	сanvas.place(height = 1500, width =1500) & Размещение canvas на окне tkinter. \\
	\hline
	BaldursGame.timer & Вызов метода timer у BaldursGame. \\
	\hline
	root.mainloop() & Основной цикл обработки событий. \\
	\hline
\end{xltabular}

\renewcommand{\arraystretch}{1.0}

\subsection{Модульное тестирование разработанного приложения}

Модульный тест для класса Rectangle из модели данных представлен на рисунке \ref{unitRec:image}.

\begin{figure}[H]
\begin{lstlisting}[language=Python]
import unittest
from rpg.rectangle import Rectangle

class TestRectangle(unittest.TestCase):
	def setUp(self):
		# Прямоугольник для использования в тестах
		self.rect = Rectangle(1, 1, 4, 4)
	
	def test_inside(self):
		'''Тест: прямоугольник внутри другого'''
		rect_outside = Rectangle(0, 0, 6, 6)
		self.assertTrue(self.rect.is_in(rect_outside))
	
	def test_outside(self):
		'''Тест: прямоугольник снаружи другого'''
		rect_inside = Rectangle(2, 2, 2, 2)
		self.assertFalse(self.rect.is_in(rect_inside))
	
	def test_apartside(self):
		'''Тест: прямоугольник отдельно от другого'''
		rect_apart = Rectangle(6, 6, 2, 2)
		self.assertFalse(self.rect.is_in(rect_apart))
	
	def test_touching_left(self):
		'''Тест: прямоугольник касается слева'''
		touching_left = Rectangle(0, 2, 1, 1)
		self.assertFalse(self.rect.is_in(touching_left))
		
	def test_touching_right(self):
		'''Тест: прямоугольник касается справа'''
		touching_right = Rectangle(5, 2, 1, 1)
		self.assertFalse(self.rect.is_in(touching_right))
	
	def test_touching_top(self):
		'''Тест: прямоугольник касается сверху'''
		touching_top = Rectangle(2, 5, 1, 1)
		self.assertFalse(self.rect.is_in(touching_top))
	
	def test_touching_bottom(self):
		'''Тест: прямоугольник касается снизу'''
		touching_bottom = Rectangle(2, 0, 1, 1)
		self.assertFalse(self.rect.is_in(touching_bottom))
	

\end{lstlisting}  
\caption{Модульный тест класса Rectangle}
\label{unitRec:image}
\end{figure}
Продолжение модульного теста для класса Rectangle из модели данных представлен на рисунке \ref{unitRec2:image}.
\begin{figure}[H]
	\begin{lstlisting}[language=Python]
			def test_intersect_left(self):
				'''Тест: пересечение прямоугольника слева'''
				intersect_left = Rectangle(0, 2, 3, 2)
				self.assertTrue(self.rect.is_in(intersect_left))
			
			def test_intersect_right(self):
				'''Тест: пересечение прямоугольника справа'''
				intersect_right = Rectangle(3, 2, 3, 2)
				self.assertTrue(self.rect.is_in(intersect_right))
			
			def test_intersect_top(self):
				'''Тест: пересечение прямоугольника сверху'''
				intersect_top = Rectangle(2, 3, 2, 3)
				self.assertTrue(self.rect.is_in(intersect_top))
			
			def test_intersect_bottom(self):
				'''Тест: пересечение прямоугольника снизу'''
				intersect_bottom = Rectangle(2, 0, 2, 3)
				self.assertTrue(self.rect.is_in(intersect_bottom))
			
			def test_is_point_inside(self):
				# Создайте прямоугольник
				rect = Rectangle(0, 0, 10, 10)
				
				# Точка внутри прямоугольника
				self.assertTrue(rect.is_point_inside(5, 5))
				
				# Точка на границе прямоугольника
				self.assertTrue(rect.is_point_inside(0, 0))
				self.assertTrue(rect.is_point_inside(0, 10))
				self.assertTrue(rect.is_point_inside(10, 0))
				self.assertTrue(rect.is_point_inside(10, 10))
				
				# Точка вне прямоугольника
				self.assertFalse(rect.is_point_inside(-1, -1))
				self.assertFalse(rect.is_point_inside(11, 11))
				self.assertFalse(rect.is_point_inside(5, -5))
				self.assertFalse(rect.is_point_inside(-5, 5))
		
		if __name__ == '__main__':
		unittest.main()
	\end{lstlisting}  
	\caption{Модульный тест класса Rectangle}
	\label{unitRec2:image}
\end{figure}

\subsection{Системное тестирование разработанного приложения}

На рисунке \ref{systemtest_recponse:image} представлен пример работы программы.
\begin{figure}[H]
	\centering
	\includegraphics[width=1\linewidth]{systemtest\_recponse}
	\caption{Пример работы программы с одним персонажем внутри одной, игровой зоны Village}
	\label{systemtest_recponse:image}
\end{figure}

На рисунке \ref{systemtest_recponse1:image} представлен пример анимации персонажа.
\begin{figure}[H]
	\centering
	\includegraphics[width=1\linewidth]{systemtest\_recponse1}
	\caption{Анимация передвижения персонажа mage}
	\label{systemtest_recponse1:image}
\end{figure}

На рисунке \ref{systemtest_recponse2:image} представлен пример движения персонажа.
\begin{figure}[H]
	\centering
	\includegraphics[width=1\linewidth]{systemtest\_recponse2}
	\caption{Передвижение персонажа mage}
	\label{systemtest_recponse2:image}
\end{figure}

На рисунке \ref{systemtest_recponse3:image} представлен пример невозможности выхода за границу зоны.
\begin{figure}[H]
	\centering
	\includegraphics[width=1\linewidth]{systemtest\_recponse3}
	\caption{Персонаж mage, не может выйти за пределы видимой зоны Village}
	\label{systemtest_recponse3:image}
\end{figure}

На рисунке \ref{systemtest_recponse4:image} представлен пример перехода персонажа из зоны.
\begin{figure}[H]
	\centering
	\includegraphics[width=1\linewidth]{systemtest\_recponse4}
	\caption{Персонаж mage, переходит из зоны Village в зону Ruins}
	\label{systemtest_recponse4:image}
\end{figure}

На рисунке \ref{systemtest_recponse5:image} представлен пример установки новой зоны.
\begin{figure}[H]
	\centering
	\includegraphics[width=1\linewidth]{systemtest\_recponse5}
	\caption{Пример работы программы с тремя персонажами внутри одной, игровой зоны Ruins}
	\label{systemtest_recponse5:image}
\end{figure}

На рисунке \ref{systemtest_recponse6:image} представлен пример работы сценария движения персонажа.
\begin{figure}[H]
	\centering
	\includegraphics[width=1\linewidth]{systemtest\_recponse6}
	\caption{Пример работы сценария walk(50, 50, self.footman), игровой зоны Ruins}
	\label{systemtest_recponse6:image}
\end{figure}

На рисунке \ref{systemtest_recponse7:image} представлен пример работы сценария поведения персонажа противника.
\begin{figure}[H]
	\centering
	\includegraphics[width=1\linewidth]{systemtest\_recponse7}
	\caption{Пример работы сценария ai(self.grunt), игровой зоны Ruins}
	\label{systemtest_recponse7:image}
\end{figure}

На рисунке \ref{systemtest_recponse8:image} представлен пример работы метода click персонажа.
\begin{figure}[H]
	\centering
	\includegraphics[width=1\linewidth]{systemtest\_recponse8}
	\caption{Вызов метода click, у персонажа Grunt}
	\label{systemtest_recponse8:image}
\end{figure}