\section{Техническое задание}
\subsection{Основание для разработки}

Основанием для разработки является задание на выпускную квалификационную работу бакалавра <"Платформа для создания компьютерных изометрических ролевых игр с заранее отрисованным двухмерным фоном и спрайтовыми персонажами">.

\subsection{Цель и назначение разработки}

Основной задачей выпускной квалификационной работы является разработка движка ролевых игр для продвижения популярности их».

 Данный программный продукт предназначен для демонстрации практических навыков, полученных в течение обучения. Исходя из этого, основную цель предлагается рассмотреть в разрезе двух групп подцелей.

Задачами данной разработки являются:
\begin{itemize}
\item создание персонажа игрока.
\item реализация генерируемых локаций.
\item реализация основных механик Dungeons and Dragons.
\item реализация сражения персонажа игрока с монстрами.
\item реализация передвижения персонажа игрока по подземелью.
\end{itemize}

\subsection{Требования пользователя к движку}

движок должен включать в себя:
\begin{itemize}
    \item генерацию локаций.
    \item генерацию персонажа.
    \item генерацию монстров.
    \item реализацию пошаговой системы боя.
    \item иммитацию основных механик ролевой игры Dungeons and Dragons.
\end{itemize}

\subsection{Правила игры}
 \begin{itemize}
 	\item ролевая игра моделирует все основные механики Dungeons and Dragons, в которой игрок управляет персонажем, который бродит по одноуровневому подземелью, собирая сокровища и убивая монстров. Подземелье визуализируется в двухмерном виде сверху с использованием экранной графики персонажей и управляется с помощью команд с клавиатуры. Подземелье имеет фиксированную планировку, но встречи с монстрами и сокровища генерируются случайным образом. Они генерируются всякий раз, когда создается новый персонаж, и сохраняются вместе с ним.
 	\begin{itemize}
 		\item 1. Цель игры:Основная цель игры заключается в исследовании мира, выполнении заданий и квестов, сражении с врагами и развитии своего персонажа. Игра также имеет главный сюжет, который игрок может прогрессировать, следуя определенным событиям и заданиям
 		\item 2. Боевая система:  Бои могут происходить в режиме реального времени или пошаговом режиме, в зависимости от настроек игры. Игрок может управлять группой персонажей и давать им команды в бою. В бою игрок может использовать различные атаки, заклинания и способности своего персонажа для победы над врагами
 		\item 3. Персонажи: Игрок может создать своего уникального персонажа, выбрав класс, расу, навыки и характеристики. Каждый класс имеет свои особенности и специализации, определяющие стиль игры и возможности персонажа. Персонажи могут повышать уровень, получать новые навыки и способности, улучшать характеристики и собирать экипировки
 		\item 4. Исследование мира: Игрок может свободно перемещаться по миру игры, исследуя различные локации и взаимодействуя с окружающими объектами. Во время исследования игрок может встретить неигровых персонажей (NPC), с которыми можно общаться, получать задания и информацию о мире.
 		\item 5. Прогрессия и развитие: Игрок может зарабатывать опыт и повышать уровень своего персонажа. Повышение уровня позволяет персонажу получать новые навыки, улучшать характеристики и получать новые способности. Игрок также может собирать и улучшать экипировку для своего персонажа, чтобы повысить его силу и выживаемость.
 		\item 6. Задания и квесты: Игрок может выполнять различные задания и квесты, предлагаемые неигровыми персонажами. Задания могут включать поиск предметов, убийство определенных врагов, решение головоломок и т.д. За выполнение заданий игрок может получать награды, опыт и продвигаться в сюжете игры.
 	\end{itemize}
 \end{itemize}
 
\subsection{Особенности Dungeons and Dragons}
\begin{itemize}
	\item Dungeons and Dragons (DnD) - это настольная ролевая игра, в которой игроки сотрудничают вместе, чтобы создать историю в фантастическом мире. В DnD один игрок выступает в роли Мастера игры (Мастера подземелий), который рассказывает и контролирует мир, а остальные игроки играют за своих персонажей, которых они создают и развивают.
	\begin{itemize} Основные элементы ролевой системы DnD включают:
		\item 1. Классы и расы: Классы представляют различные роли и специализации персонажей, такие как воин, маг, жрец. Каждый класс имеет свои уникальные способности и навыки.
			\begin{itemize}
				\item Особенности воина: воин специализируется на ближнем бою, может использовать все виды оружия, может носить все доспехи и щиты, не способен накладывать заклинания, его кость здоровья 10-гранный кубик (D10).
				\item Особенности мага: маг специализируется на дальнем бою, может использовать только боевые посохи и короткие мечи, не может носить доспехи , способен накладывать заклинания, наносящие большое количество урона, его кость здоровья 6-гранный кубик (D6).
				\item Особенности жреца: жрец специализируется на ближнем бою, может использовать простое оружия, может носить лёгкие,средние доспехи и щиты, способен накладывать заклинания, исцеляющие его, его кость здоровья 8-гранный кубик (D8).
			\end{itemize}

		\item Расы определяют происхождение персонажа и дают особые характеристики и способности. Примеры рас включают эльфов, дварфов, людей.
			\begin{itemize}
			\item Особенности человека: человек на старте получает +1 ко всем характеристикам, его базовая скорость передвижения равна 30 футам, его размер средний.
			\item Особенности эльфа: эльф получает +2 к ловкости и +1 к мудрости, его базовая скорость передвижения равна 35 футов, его размер средний, у эльфа есть тёмное зрение в радиусе 30 футов.
			\item Особенности дварфа: дварф получает +2 к силе и +2 к телосложению, его базовая скорость передвижения равна 25 футов, его размер маленький, у дварфа есть тёмное зрение в радиусе 30 футов.
			\end{itemize}
		\item 2. Характеристики:
		\begin{itemize}
			\item Характеристики определяют физические и умственные способности персонажа, такие как сила, ловкость, телосложение, интеллект, мудрость, харизма. Они влияют на способности и успех персонажа в различных ситуациях.
			\item Сила - характеристика влияющая на броски атак рукопашным оружием, а так же на проверки навыков: атлетика.
			\item Ловкость - характеристика влияющая на броски атак совершаемых стрелковым оружием, на класс доспеха персонажа, а так же на проверки навыков: акробатика, ловкость рук, скрытность.
			\item Телосложение - характеристика влияющая на колличество здоровья персонажа.
			\item Интеллект - характеристика влияющая на броски атак совершённых заклинаниями волшебника, а так же на проверки навыков: магия, история, природа, расследование, религия.
			\item Мудрость - характеристика влияющая на броски атак  совершённых заклинаниями жреца, а так же на проверки навыков: восприятие, выживание, проницательность, уход за животными, медицина.
			\item Харизма - характеристика влияющая на общение с не игровыми персонажами, а так же на проверки навыков: выступление, убеждение, обман, запугивание.
		\end{itemize}
		\item 3. Навыки:
		\begin{itemize}
			\item Навыки представляют специализации персонажа в определенных областях, таких как взлом замков, обращение с оружием, магия и т.д. Навыки могут быть использованы для выполнения действий и решения задач
		\end{itemize}
		\item 4. Броски костей:
		\begin{itemize}
			\item Игра DnD использует различные виды игровых костей для случайной генерации результатов. Например, для определения успеха атаки или проверки навыка игрок может бросить 20-гранный кубик (D20) и добавить соответствующие модификаторы.
		\end{itemize}
		\item 5. Приключения и задания:
		\begin{itemize}
			\item Мастер игры создает историю, включающую задания и приключения, которые игроки выполняют. Задания могут включать исследование подземелий, сражение с монстрами, решение головоломок и взаимодействие с неигровыми персонажами.
		\end{itemize}
		\item 6: Прогрессия и опыт:
		\begin{itemize}
			\item Персонажи получают опыт за выполнение заданий и сражение с врагами. Зарабатывая опыт, персонажи повышают уровень, получают новые способности и становятся сильнее.
		\end{itemize}
		\item 7. Магия:
		\begin{itemize}
			\item DnD имеет разветвленную систему магии, позволяющую персонажам использовать заклинания различных уровней и школ. Магические заклинания могут влиять на бой, лечение, обнаружение и другие аспекты игры.
		\end{itemize}
	\end{itemize}
\end{itemize}
\subsection{Игровой мир Фаэрун}
Континент включает в себя самые разнообразные территории. Помимо береговых линий на западе и на юге, основной особенностью континента является Море Падающих Звезд. Это несимметричное море, которое орошает внутренние земли и соединяет западные и восточные регионы Фаэруна, а также является главным торговым маршрутом для многих наций
Регион дикой местности, тяжелых погодных условий, орд орков и диких варварских племен, В основном регион называют просто "Север", также имея в виду северную часть Побережья Мечей. Побережье Мечей пролегает вдоль Моря Мечей, от северной границы Амна до Моря Движущегося Льда, преимущественно занимается городами-государствами, использующими море в торговых целях. Границами региона обычно считают города Невервинтер на севере и Врата Балдура на юге, но и земли к северу и югу от них, не находящиеся под контролем каких-либо более влиятельных сил, часто тоже включаются в карты Побережья Мечей. Регион Севера при этом, являясь более широкой географической областью, включает в себя всё к северу от Амна, и разделяется на два основных региона: Западное Сердцеземье и Дикую Границу. Западное Сердцеземье включает в себя узкую полосу цивилизации между Горами Заката и Морем Мечей, и к северу, от Тролльих Гор и Облачных Пиков до Торгового Пути. К Дикой Границе относится весь остальной Север, состоящий из совсем незаселенных либо скудно заселенных земель, не включая крупные города и различные мелкие поселения, находящиеся в их непосредственной сфере влияния. Большинство поселений, наций и государств Севера могут быть отнесены к одной из пяти категорий: члены Альянса Лордов, дварфские крепости, островные государства, независимые королевства, разбросанные по побережью и глубины Подземья. По большей части это дикие земли и неисследованные земли, лежащие между большой пустыней Анаурох на востоке и крупным регионом Побережья Мечей на западе, южной границей которых считается Высокая Пустошь.

\subsection{монстры мира Фаэрун}

	\begin{itemize}
		\item Скелет
		\item тип существа: нежить, размер: средний.
		\item показатель опасности 1/4(50 опыта).
		\item класс доспеха 12, здоровье 13 единиц.
		\item скорость 30 футов.
		\item характеристики: СИЛ 10(+0) ЛОВ 14(+2) ТЕЛ 15(+2) ИНТ 6(-2) МУД 8(-1) ХАР 5(-3).
		\item уязвимость к урону: дробящий.
		\item иммунитет к урону: яд.
		\item чувства: пассивное восприятие 9.	
		\begin{itemize}
			\item 1.1 Действия:
			\item Короткий меч. Рукопашная атака оружием: +4 к попаданию, досягаемость 5 футов, одна цель. Попадание: Колющий урон 5 (1к6 + 2).
		\end{itemize}
	\end{itemize}
	\begin{itemize}
		\item Людоящер
		\item тип существа: гуманоид, размер: средний.
		\item показатель опасности 1/2(100 опыта).
		\item класс доспеха 14, здоровье 22 единиц.
		\item скорость 30 футов.
		\item характеристики: СИЛ 15(+2) ЛОВ 10(+0) ТЕЛ 13(+1) ИНТ 7(-2) МУД 12(+1) ХАР 7(-2).
		\item иммунитет к урону: яд.
		\item чувства: пассивное восприятие 11.
		\begin{itemize}
			\item 2.1 Действия: Мультиатака. Людоящер совершает две рукопашные атаки.
			\item Укус. Рукопашная атака оружием: +4 к попаданию, досягаемость 5 футов, одна цель. Попадание: Колющий урон 5 (1к6 + 2).
		\end{itemize}
	\end{itemize}
	\begin{itemize}
		\item Бурый медведь
		\item тип существа: Зверь, размер: Большой.
		\item показатель опасности 1(200 опыта).
		\item класс доспеха 11, здоровье 34 единиц.
		\item скорость 40 футов.
		\item характеристики: СИЛ 19(+4) ЛОВ 10(+0) ТЕЛ 16(+3) ИНТ 2(-4) МУД 13(+1) ХАР 7(-2).
		\item чувства: пассивное восприятие 11.
		\begin{itemize}
			\item 3.1 Действия: Мультиатака. Медведь совершает две атаки: одну укусом, и одну когтями.
			\item Укус. Рукопашная атака оружием: +6 к попаданию, досягаемость 5 футов, одна цель. Попадание: Колющий урон 8 (1к8 + 4).
			\item Когти. Рукопашная атака оружием: +6 к попаданию, досягаемость 5 футов, одна цель. Попадание: Рубящий урон 11 (2к6 + 4).
		\end{itemize}
	\end{itemize}
	\begin{itemize}
		\item Огр
		\item тип существа: монстр, размер: Большой.
		\item показатель опасности 2(450 опыта).
		\item класс доспеха 11, здоровье 59 единиц.
		\item скорость 40 футов.
		\item характеристики: СИЛ 19(+4) ЛОВ 8(-1) ТЕЛ 16(+3) ИНТ 5(-3) МУД 7(-2) ХАР 7(-2).
		\item чувства:тёмное зрение 60 футов, пассивное восприятие 8.
		\begin{itemize}
			\item 4.1 Действия:
			\item Палица. Рукопашная атака оружием: +6 к попаданию, досягаемость 5 футов, одна цель. Попадание: 13 (2к8 + 4) дробящего урона.
		\end{itemize}
	\end{itemize}


\subsection{Интерфейс пользователя}
\begin{itemize}
	\item 1. Генерация персонажа. Пользователь видит перед собой окно с кнопкой "Создать персонажа". Кликнув на неё, на экране появятся лейбл с текстом "Выбирете расу" и 3 кнопки с названиями: "Человек","Эльф", "Дварф". После
	\item 2. Интерфейс пользователя будет состоять из окна, на котором будет отображаться персонаж и локация, на которой он находится. В левом нижнем углу окна будет находится красная полоска отображающая здоровье персонажа. Справа от неё будет находится кнопка с названием "Инвентарь", нажав на которую откроется новое окно "Инвентаря". В нём будет список предметов которыми сейчас владеет персонаж. Справа от кнопки "инвентаря" будет находится кнопка "Заклинания". Кликнув на неё открывается окно, в котором будет представлен список всех известных на данный момент заклинаний, а так же количество ячеек, которые персонаж может израсходовать, на применение заклинаний. Справа от кнопки "Заклинания", будет находится кнопка "Особенности и черты". Кликнув на неё откроется окно, в котором в виде списка будут перечислены все особенности расы и черты класса выбранные пользователем, а так же уровень персонажа. Справа от клавиши "Особенности и черты" будут находится 4 кнопки: "Взаимодействовать", "Атаковать", "Сотворить заклинание", "Добавить".
	\item 3. Пользователь будет мышкой водить по видимым объектам, на экране. Наводя мышь на объект и нажав ЛКМ, пользователь сможет с ним взаимодействовать. Если это объект окружения, то пользователь может нажать на кнопку "Взаимодействовать", и откроется окно взаимодействия с объектом. Если это монстр, то  пользователь может нажать на кнопку "Атаковать" или "Сотворить заклинание". В случае "Атака" откроется окно инвентаря, и пользователь должен будет выбрать каким оружием он совершит атаку. В случае "Сотворить заклинание" откроется окно "Заклинания", и пользователю будет нужно выбрать одно из списка заклинаний. Если это НПС, то пользователь может нажать кнопку "Взаимодействовать", и откроется окно взаимодействия с НПС.
	\item 4. Клавишами WASD пользователь будет передвигать игрового персонажа: вверх, влево, вниз, вправо.
\end{itemize}
\subsection{Пример игры}
Игрок создаёт своего персонажа, выбирая один из возможных классов воин, волшебник, следопыт и т.д. Затем он выбирает расу своему персонажу: человек, дварф, эльф и т.д. Так игрок создал своего человека воина 1 уровня. После этого персонаж появляется в стартовой локации. В ней находится NPC: человек-страж по имени Даниэль, гном-торговец Владимир. Игрок может подойти к Даниэлю и кликнуть по нему, чтобы начать диалог, на экран выведется сообщение со следующим текстом: "Исследователь подземелий, мы ждали тебя, это одно из многих подземелий Ацецерака. Этот могучий лич хранит в этом месте множество могучих артефактов и горы несметных сокровищ. По приказу короля Персиваля Дэ Ролло третьего, мы созываем искателей приключений, чтобы очистить это место от бродящих по нему монстров. Искатель здесь ты можешь отдохнуть, перед входом в подземелье, так же ты можешь покупать и продавать драгоценности и вещи у Владимира. Вот он". После этого у игрока появляются квесты: 1) "пообщаться с Владимиром" 2) "Первые изучения". Игрок передвигает персонажа к гному, кликает по нему, на экран выведется сообщение со следующим текстом:"Здравствуй, дорогой друг, я Владимир местный торговец, у меня ты можешь купить снаряжение или продать предметы найденные в подземелье". После этого квест "Пообщаться с Владимиром" будет выполнен. Персонаж получит за это 25 опыта. После игрок нажмёт на левой кнопкой мыши на иконку входа в подземелье. Персонаж игрока перенесётся внутрь подземелья. В этой локации будет пустая комната с двумя проходами, игрок подойдёт к проходу номер 1, персонаж переместился в новую локацию(комнату). В ней находится скелет, начинается боевая ситуация. Игровая система бросает инициативу за скелета и персонажа игрока, первым ходит персонаж игрока. Игрок перемещает персонажа левым щелчком мыши на 6 клеток к скелету, а затем нажимает на кнопку атака мечом. Боевая система делает скрытый бросок на атаку мечом по скелету, попадание, боевая система делает бросок на урон длинным мечом, 11 рубящего урона, здоровье скелета падает до нуля. Боевая ситуация заканчивается. Игрок нажимает левой кнопкой мыши на кости побеждённого скелета. К персонажу в инвентарь, добавляется "ржавый короткий меч". После этого игрок замечает 2 прохода в этой комнате, проход номер 1, тот через который игрок прошёл в эту комнату, проход номер 2 ведёт в неизвестную комнату. Так же квест "Первые изучения" считается выполненным. Персонаж игрока получает 50 опыта за победу над скелетом, и ещё 100 опыта за выполнение квеста. Игрок перемещает персонажа вправо, а затем кликает на проход номер 1, после чего перемещается в первую комнату. Затем подходит к проходу номер 2, кликает по нему и выходит из подземелья. Когла он оказывается на стартовой локации, игрок кликает по Даниэлю и тот выдаёт ему квест "Убейте трёх скелетов в подземелье".
\subsection{ Моделирование вариантов использования}
На основании анализа предметной области в программе должны быть реализованы следующие прецеденты:
\begin{enumerate}
\item Создание персонажа.
\item Боевая система.
\item Развитие персонажа.
\item Исследование мира.
\end{enumerate}

\subsection{Требования к оформлению документации}

Разработка программной документации и программного изделия должна производиться согласно ГОСТ 19.102-77 и ГОСТ 34.601-90. Единая система программной документации.
