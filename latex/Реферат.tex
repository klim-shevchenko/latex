\abstract{РЕФЕРАТ}

Объем работы равен \formbytotal{lastpage}{страниц}{е}{ам}{ам}. Работа содержит \formbytotal{figurecnt}{иллюстраци}{ю}{и}{й}, \formbytotal{tablecnt}{таблиц}{у}{ы}{}, \arabic{bibcount} библиографических источников и \formbytotal{числоПлакатов}{лист}{}{а}{ов} графического материала. Количество приложений – 2. Графический материал представлен в приложении А. Фрагменты исходного кода представлены в приложении Б.

Перечень ключевых слов: платформа, система, игра, РПГ, Python, сценарии, скрипты, многопоточность, изображения, информатизация, автоматизация, информационные технологии, спрайт,  программное обеспечение, классы, обработка клика мыши, подсистема, компонент, модуль, сущность, информационный блок, метод, разработчик, геймдизайнер, пользователь.

Объектом разработки является платформа для создания компьютерных изометрических ролевых игр с заранее отрисованным двумерным фоном и спрайтовыми персонажами.

Целью выпускной квалификационной работы является разработка приложения для разработки компьютерных ролевых игр с заранее отрисованными спрайтами и фоном.

В процессе создания приложения были выделены основные сущности путем создания информационных блоков, использованы классы и методы модулей, обеспечивающие работу с сущностями предметной области, а также корректную работу приложения для разработки рпг-игр, разработаны разделы, содержащие информацию о рпг-играх, игровых платформах для создания игр, графике, языке программирования Python, используемых библиотеках tkinter, treading.

\selectlanguage{english}
\abstract{ABSTRACT}
  
The volume of work is \formbytotal{lastpage}{page}{}{s}{s}. The work contains \formbytotal{figurecnt}{illustration}{}{s}{s}, \formbytotal{tablecnt}{table}{}{s}{s}, \arabic{bibcount} bibliographic sources and \formbytotal{числоПлакатов}{sheet}{}{s}{s} of graphic material. The number of applications is 2. The graphic material is presented in annex A. The layout of the site, including the connection of components, is presented in annex B.

List of keywords: platform, system, game, RPG, Python, scenarios, scripts, multithreading, images, information, automation, information technology, sprite, software, classes, mouse click processing, subsystem, component, module, entity, information block , method, developer, game designer, user.

The object of development is a platform for creating computer isometric role-playing games with pre-rendered two-dimensional backgrounds and sprite characters.

The purpose of the final qualifying work is to popularize RPG games.

In the process of creating the application, the main entities were identified by creating information blocks, classes and methods of modules were used to ensure work with entities of the subject area, as well as the correct operation of the application for developing RPG games, sections were developed containing information about RPG games, gaming platforms for game creation, graphics, Python programming language, tkinter, treading libraries used.

\selectlanguage{russian}
