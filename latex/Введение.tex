\section*{ВВЕДЕНИЕ}
\addcontentsline{toc}{section}{ВВЕДЕНИЕ}

С развитием цифровых технологий и увеличением вычислительной мощности персональных компьютеров, появилась возможность создания сложных и многофункциональных программных продуктов, в том числе и для развлекательной индустрии. Одним из таких направлений является разработка компьютерных ролевых игр (RPG), которые погружают пользователя в виртуальные миры с заранее отрисованными фонами и спрайтами. Эти элементы игры не только создают уникальную атмосферу и мир, но и являются ключевыми компонентами в структуре игрового процесса.

Как и аддитивные технологии, которые кардинально изменили подход к проектированию и производству, платформы для создания RPG представляют собой инновационный инструмент, который позволяет разработчикам с минимальными затратами времени и ресурсов создавать захватывающие игры. Это стало возможным благодаря использованию готовых ассетов, таких как фоны и спрайты, а также благодаря гибким инструментам для их интеграции и анимации.

Таким образом, платформы для создания RPG игр с заранее отрисованным фоном и спрайтами являются частью более широкого тренда цифровизации и автоматизации, который охватывает многие отрасли, включая развлекательную индустрию. Они позволяют разработчикам сосредоточиться на творческом процессе, минимизируя технические аспекты реализации проекта.

\emph{Цель настоящей работы} – разработка приложения для разработки компьютерных ролевых игр с заранее отрисованными спрайтами и фоном. Для достижения поставленной цели необходимо решить \emph{следующие задачи:}
\begin{itemize}
\item провести анализ предметной области;
\item разработать концептуальную модель приложения;
\item спроектировать приложение;
\item реализовать приложение средствами языка программирования python.
\end{itemize}

\emph{Структура и объем работы.} Отчет состоит из введения, 4 разделов основной части, заключения, списка использованных источников, 2 приложений. Текст выпускной квалификационной работы равен \formbytotal{page}{страниц}{е}{ам}{ам}.

\emph{Во введении} сформулирована цель работы, поставлены задачи разработки, описана структура работы, приведено краткое содержание каждого из разделов.

\emph{В первом разделе} на стадии описания технической характеристики предметной области приводится сбор информации о деятельности компании, для которой осуществляется разработка сайта.

\emph{Во втором разделе} на стадии технического задания приводятся требования к разрабатываемому приложению.

\emph{В третьем разделе} на стадии технического проектирования представлены проектные решения для приложения.

\emph{В четвертом разделе} приводится список классов и их методов, использованных при разработке сайта, производится тестирование разработанного приложения.

В заключении излагаются основные результаты работы, полученные в ходе разработки.

В приложении А представлен графический материал.
В приложении Б представлены фрагменты исходного кода. 
